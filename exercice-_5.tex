% Options for packages loaded elsewhere
\PassOptionsToPackage{unicode}{hyperref}
\PassOptionsToPackage{hyphens}{url}
%
\documentclass[
]{article}
\usepackage{lmodern}
\usepackage{amssymb,amsmath}
\usepackage{ifxetex,ifluatex}
\ifnum 0\ifxetex 1\fi\ifluatex 1\fi=0 % if pdftex
  \usepackage[T1]{fontenc}
  \usepackage[utf8]{inputenc}
  \usepackage{textcomp} % provide euro and other symbols
\else % if luatex or xetex
  \usepackage{unicode-math}
  \defaultfontfeatures{Scale=MatchLowercase}
  \defaultfontfeatures[\rmfamily]{Ligatures=TeX,Scale=1}
\fi
% Use upquote if available, for straight quotes in verbatim environments
\IfFileExists{upquote.sty}{\usepackage{upquote}}{}
\IfFileExists{microtype.sty}{% use microtype if available
  \usepackage[]{microtype}
  \UseMicrotypeSet[protrusion]{basicmath} % disable protrusion for tt fonts
}{}
\makeatletter
\@ifundefined{KOMAClassName}{% if non-KOMA class
  \IfFileExists{parskip.sty}{%
    \usepackage{parskip}
  }{% else
    \setlength{\parindent}{0pt}
    \setlength{\parskip}{6pt plus 2pt minus 1pt}}
}{% if KOMA class
  \KOMAoptions{parskip=half}}
\makeatother
\usepackage{xcolor}
\IfFileExists{xurl.sty}{\usepackage{xurl}}{} % add URL line breaks if available
\IfFileExists{bookmark.sty}{\usepackage{bookmark}}{\usepackage{hyperref}}
\hypersetup{
  pdftitle={Séries temporelles et reproductibilité},
  pdfauthor={Andrea Katherín Carranza Díaz},
  hidelinks,
  pdfcreator={LaTeX via pandoc}}
\urlstyle{same} % disable monospaced font for URLs
\usepackage[margin=1in]{geometry}
\usepackage{color}
\usepackage{fancyvrb}
\newcommand{\VerbBar}{|}
\newcommand{\VERB}{\Verb[commandchars=\\\{\}]}
\DefineVerbatimEnvironment{Highlighting}{Verbatim}{commandchars=\\\{\}}
% Add ',fontsize=\small' for more characters per line
\usepackage{framed}
\definecolor{shadecolor}{RGB}{248,248,248}
\newenvironment{Shaded}{\begin{snugshade}}{\end{snugshade}}
\newcommand{\AlertTok}[1]{\textcolor[rgb]{0.94,0.16,0.16}{#1}}
\newcommand{\AnnotationTok}[1]{\textcolor[rgb]{0.56,0.35,0.01}{\textbf{\textit{#1}}}}
\newcommand{\AttributeTok}[1]{\textcolor[rgb]{0.77,0.63,0.00}{#1}}
\newcommand{\BaseNTok}[1]{\textcolor[rgb]{0.00,0.00,0.81}{#1}}
\newcommand{\BuiltInTok}[1]{#1}
\newcommand{\CharTok}[1]{\textcolor[rgb]{0.31,0.60,0.02}{#1}}
\newcommand{\CommentTok}[1]{\textcolor[rgb]{0.56,0.35,0.01}{\textit{#1}}}
\newcommand{\CommentVarTok}[1]{\textcolor[rgb]{0.56,0.35,0.01}{\textbf{\textit{#1}}}}
\newcommand{\ConstantTok}[1]{\textcolor[rgb]{0.00,0.00,0.00}{#1}}
\newcommand{\ControlFlowTok}[1]{\textcolor[rgb]{0.13,0.29,0.53}{\textbf{#1}}}
\newcommand{\DataTypeTok}[1]{\textcolor[rgb]{0.13,0.29,0.53}{#1}}
\newcommand{\DecValTok}[1]{\textcolor[rgb]{0.00,0.00,0.81}{#1}}
\newcommand{\DocumentationTok}[1]{\textcolor[rgb]{0.56,0.35,0.01}{\textbf{\textit{#1}}}}
\newcommand{\ErrorTok}[1]{\textcolor[rgb]{0.64,0.00,0.00}{\textbf{#1}}}
\newcommand{\ExtensionTok}[1]{#1}
\newcommand{\FloatTok}[1]{\textcolor[rgb]{0.00,0.00,0.81}{#1}}
\newcommand{\FunctionTok}[1]{\textcolor[rgb]{0.00,0.00,0.00}{#1}}
\newcommand{\ImportTok}[1]{#1}
\newcommand{\InformationTok}[1]{\textcolor[rgb]{0.56,0.35,0.01}{\textbf{\textit{#1}}}}
\newcommand{\KeywordTok}[1]{\textcolor[rgb]{0.13,0.29,0.53}{\textbf{#1}}}
\newcommand{\NormalTok}[1]{#1}
\newcommand{\OperatorTok}[1]{\textcolor[rgb]{0.81,0.36,0.00}{\textbf{#1}}}
\newcommand{\OtherTok}[1]{\textcolor[rgb]{0.56,0.35,0.01}{#1}}
\newcommand{\PreprocessorTok}[1]{\textcolor[rgb]{0.56,0.35,0.01}{\textit{#1}}}
\newcommand{\RegionMarkerTok}[1]{#1}
\newcommand{\SpecialCharTok}[1]{\textcolor[rgb]{0.00,0.00,0.00}{#1}}
\newcommand{\SpecialStringTok}[1]{\textcolor[rgb]{0.31,0.60,0.02}{#1}}
\newcommand{\StringTok}[1]{\textcolor[rgb]{0.31,0.60,0.02}{#1}}
\newcommand{\VariableTok}[1]{\textcolor[rgb]{0.00,0.00,0.00}{#1}}
\newcommand{\VerbatimStringTok}[1]{\textcolor[rgb]{0.31,0.60,0.02}{#1}}
\newcommand{\WarningTok}[1]{\textcolor[rgb]{0.56,0.35,0.01}{\textbf{\textit{#1}}}}
\usepackage{graphicx,grffile}
\makeatletter
\def\maxwidth{\ifdim\Gin@nat@width>\linewidth\linewidth\else\Gin@nat@width\fi}
\def\maxheight{\ifdim\Gin@nat@height>\textheight\textheight\else\Gin@nat@height\fi}
\makeatother
% Scale images if necessary, so that they will not overflow the page
% margins by default, and it is still possible to overwrite the defaults
% using explicit options in \includegraphics[width, height, ...]{}
\setkeys{Gin}{width=\maxwidth,height=\maxheight,keepaspectratio}
% Set default figure placement to htbp
\makeatletter
\def\fps@figure{htbp}
\makeatother
\setlength{\emergencystretch}{3em} % prevent overfull lines
\providecommand{\tightlist}{%
  \setlength{\itemsep}{0pt}\setlength{\parskip}{0pt}}
\setcounter{secnumdepth}{-\maxdimen} % remove section numbering

\title{Séries temporelles et reproductibilité}
\author{Andrea Katherín Carranza Díaz}
\date{2020-07-29}

\begin{document}
\maketitle

\hypertarget{consigne}{%
\subsection{Consigne}\label{consigne}}

Les données du fichier hawai.csv comprennent les moyennes des mesures
mensuelles de CO2 atmosphérique entre en ppm-volume collectées au Mauna
Loa Observatory à Hawaii de mars 1958 à décembre 2001, inclusivement.

\hypertarget{lecture-des-donnuxe9es}{%
\subsection{Lecture des données}\label{lecture-des-donnuxe9es}}

\begin{Shaded}
\begin{Highlighting}[]
\KeywordTok{library}\NormalTok{(}\StringTok{"tidyverse"}\NormalTok{)}
\end{Highlighting}
\end{Shaded}

\begin{verbatim}
## -- Attaching packages --------------------------------------- tidyverse 1.3.0 --
\end{verbatim}

\begin{verbatim}
## v ggplot2 3.3.0     v purrr   0.3.4
## v tibble  3.0.1     v dplyr   0.8.5
## v tidyr   1.0.3     v stringr 1.4.0
## v readr   1.3.1     v forcats 0.5.0
\end{verbatim}

\begin{verbatim}
## -- Conflicts ------------------------------------------ tidyverse_conflicts() --
## x dplyr::filter() masks stats::filter()
## x dplyr::lag()    masks stats::lag()
\end{verbatim}

\begin{Shaded}
\begin{Highlighting}[]
\KeywordTok{library}\NormalTok{(}\StringTok{"lubridate"}\NormalTok{)}
\end{Highlighting}
\end{Shaded}

\begin{verbatim}
## 
## Attaching package: 'lubridate'
\end{verbatim}

\begin{verbatim}
## The following objects are masked from 'package:dplyr':
## 
##     intersect, setdiff, union
\end{verbatim}

\begin{verbatim}
## The following objects are masked from 'package:base':
## 
##     date, intersect, setdiff, union
\end{verbatim}

\begin{Shaded}
\begin{Highlighting}[]
\KeywordTok{library}\NormalTok{(}\StringTok{"forecast"}\NormalTok{)}
\end{Highlighting}
\end{Shaded}

\begin{verbatim}
## Warning: package 'forecast' was built under R version 4.0.2
\end{verbatim}

\begin{verbatim}
## Registered S3 method overwritten by 'quantmod':
##   method            from
##   as.zoo.data.frame zoo
\end{verbatim}

\begin{Shaded}
\begin{Highlighting}[]
\KeywordTok{library}\NormalTok{(}\StringTok{"fpp2"}\NormalTok{)}
\end{Highlighting}
\end{Shaded}

\begin{verbatim}
## Warning: package 'fpp2' was built under R version 4.0.2
\end{verbatim}

\begin{verbatim}
## Loading required package: fma
\end{verbatim}

\begin{verbatim}
## Warning: package 'fma' was built under R version 4.0.2
\end{verbatim}

\begin{verbatim}
## Loading required package: expsmooth
\end{verbatim}

\begin{verbatim}
## Warning: package 'expsmooth' was built under R version 4.0.2
\end{verbatim}

\begin{Shaded}
\begin{Highlighting}[]
\NormalTok{hawai <-}\StringTok{ }\KeywordTok{read_csv}\NormalTok{(}\StringTok{"hawai.csv"}\NormalTok{)}
\end{Highlighting}
\end{Shaded}

\begin{verbatim}
## Parsed with column specification:
## cols(
##   time = col_double(),
##   CO2 = col_double()
## )
\end{verbatim}

\begin{Shaded}
\begin{Highlighting}[]
\KeywordTok{glimpse}\NormalTok{(hawai)}
\end{Highlighting}
\end{Shaded}

\begin{verbatim}
## Rows: 526
## Columns: 2
## $ time <dbl> 1958.167, 1958.250, 1958.333, 1958.417, 1958.500, 1958.583, 19...
## $ CO2  <dbl> 316.1000, 317.2000, 317.4333, 317.4333, 315.6250, 314.9500, 31...
\end{verbatim}

Le tableau de données est lu, qui comporte 526 observations et 2
variables, toutes deux à double caractère, il est donc nécessaire de
convertir la variable \textbf{time} pour qu'elle soit reconnue comme une
date. Ceci est fait au moyen de la fonction
\texttt{lubridate::date\_decimal()} et UTC est défini comme le fuseau
horaire requis. Il est possible de vérifier cette transformation avec la
fonction de base class().

\begin{Shaded}
\begin{Highlighting}[]
\NormalTok{CO2_hawai <-}\StringTok{ }\NormalTok{hawai }\OperatorTok
\StringTok{  }\KeywordTok{mutate}\NormalTok{(}\DataTypeTok{time =} \KeywordTok{date_decimal}\NormalTok{(hawai}\OperatorTok{$}\NormalTok{time, }\DataTypeTok{tz =} \StringTok{"UTC"}\NormalTok{))}
\KeywordTok{glimpse}\NormalTok{(CO2_hawai)}
\end{Highlighting}
\end{Shaded}

\begin{verbatim}
## Rows: 526
## Columns: 2
## $ time <dttm> 1958-03-02 20:00:01, 1958-04-02 06:00:00, 1958-05-02 16:00:00...
## $ CO2  <dbl> 316.1000, 317.2000, 317.4333, 317.4333, 315.6250, 314.9500, 31...
\end{verbatim}

\begin{Shaded}
\begin{Highlighting}[]
\NormalTok{CO2_hawai }\OperatorTok\StringTok{ }\KeywordTok{pull}\NormalTok{(time) }\OperatorTok\StringTok{ }\KeywordTok{class}\NormalTok{()}
\end{Highlighting}
\end{Shaded}

\begin{verbatim}
## [1] "POSIXct" "POSIXt"
\end{verbatim}

Une fois la transformation effectuée, nous procédons à la graduation
pour observer comment le CO2 se comporte dans le temps. On observe que
les données présentent une tendance croissante et présentent également
une fluctuation saisonnière.

\begin{Shaded}
\begin{Highlighting}[]
\NormalTok{CO2_hawai }\OperatorTok
\StringTok{  }\KeywordTok{ggplot}\NormalTok{(}\KeywordTok{aes}\NormalTok{(}\DataTypeTok{x =}\NormalTok{ time, }\DataTypeTok{y =}\NormalTok{ CO2)) }\OperatorTok{+}
\StringTok{  }\KeywordTok{geom_line}\NormalTok{()}
\end{Highlighting}
\end{Shaded}

\includegraphics{exercice-_5_files/figure-latex/unnamed-chunk-3-1.pdf}

\hypertarget{cruxe9ation-de-la-suxe9rie-temporelle}{%
\subsection{Création de la série
temporelle}\label{cruxe9ation-de-la-suxe9rie-temporelle}}

La fonction \texttt{stats::ts()} sera utilisée pour créer la série
temporelle. L'argument \texttt{start}, définit la date de la première
observation (mars 1958) et l'argument \texttt{frequency} définit le
nombre d'observations par unité de temps. Dans ce cas, les mesures ont
été prises chaque mois, \texttt{frequency} = 12.

\begin{Shaded}
\begin{Highlighting}[]
\NormalTok{hawai_ts <-}\StringTok{ }\KeywordTok{ts}\NormalTok{(CO2_hawai }\OperatorTok\StringTok{ }\NormalTok{dplyr}\OperatorTok{::}\KeywordTok{select}\NormalTok{(}\OperatorTok{-}\NormalTok{time),}
             \DataTypeTok{start =} \KeywordTok{c}\NormalTok{(}\DecValTok{1958}\NormalTok{, }\DecValTok{3}\NormalTok{), }\DataTypeTok{frequency =} \DecValTok{12}\NormalTok{)}
\end{Highlighting}
\end{Shaded}

\hypertarget{suxe9paration-de-la-suxe9rie-en-parties-dentrauxeenement}{%
\subsection{Séparation de la série en parties
d'entraînement}\label{suxe9paration-de-la-suxe9rie-en-parties-dentrauxeenement}}

La fonction \texttt{windows()} sera utilisée pour séparer les séries en
parties d'entraînement et de test.

\begin{Shaded}
\begin{Highlighting}[]
\NormalTok{hawai_train <-}\StringTok{ }\KeywordTok{window}\NormalTok{(hawai_ts, }\DataTypeTok{start =} \KeywordTok{c}\NormalTok{(}\DecValTok{1958}\NormalTok{, }\DecValTok{3}\NormalTok{), }\DataTypeTok{end =} \KeywordTok{c}\NormalTok{(}\DecValTok{1988}\NormalTok{, }\DecValTok{12}\NormalTok{))}
\NormalTok{hawai_test <-}\StringTok{ }\KeywordTok{window}\NormalTok{(hawai_ts, }\DataTypeTok{start =} \KeywordTok{c}\NormalTok{(}\DecValTok{1989}\NormalTok{, }\DecValTok{1}\NormalTok{), }\DataTypeTok{end =} \KeywordTok{c}\NormalTok{(}\DecValTok{2001}\NormalTok{, }\DecValTok{12}\NormalTok{))}
\end{Highlighting}
\end{Shaded}

Environ 70 \% des mesures sont attribuées à la série d'entraînement (les
quelque 370 premières observations, qui ont été faites de mars 1958 à
décembre 1988) et les 30 \% restants des observations sont attribuées à
la série d'essai (de janvier 1989 à décembre 2001) :

\hypertarget{cruxe9ation-dun-moduxe8le-pruxe9visionnel-sur-les-donnuxe9es-dentrauxeenement}{%
\subsection{Création d'un modèle prévisionnel sur les données
d'entraînement}\label{cruxe9ation-dun-moduxe8le-pruxe9visionnel-sur-les-donnuxe9es-dentrauxeenement}}

Le modèle choisi pour les données sur la formation était le modèle
\texttt{ETS} (\emph{erreur:} sans tendance - tendance additive -
tendance adouci, \emph{tendance :} sans saison - saison additive -
saison multiplicative et \emph{saisonnalité :} erreur additive - erreur
multiplicative), qui fait partie de la famille \texttt{SES}.
L'optimisation du modèle est appliquée aux séries de formation en
utilisant la fonction de \texttt{forecast::ets()} qui permet de
connaître le type d'erreur du modèle, la tendance et la station, ainsi
que de trouver les paramètres de lissage α, β, γ du modèle. Le modèle
obtenue correspond à un modèle ETS(A,A,A). Cela signifie qu'une erreur,
une tendance et une saison de type additive sont obtenues.

la prévision est réalisée par modélisation additive en fonction du
niveau, de la tendance et de la saison. Le paramètre α décrit la
distribution des poids. Dans ce cas, on a obtenu un α=0.6698, ce qui
signifie que les événements plus récents du modèle ont un poids plus
important dans la prévision.Pendant que, le valeur de β = 0.003
(s'approche de zéro) indique que la tendance change à une faible
vitesse. γ correspond à la portion saisonnière. Généralement, ce
paramètre fluctue autour de zéro, dans ce cas, γ = 2e-04.Le modèle ne
présente pas de valeur ϕ, qui adouci la pente. Il est donc probable que
l'adoucissement ne se justifie pas.

Le graphique du niveau en fonction du temps indique que la tendance de
la concentration de CO2 dans l'atmosphère augmente avec le temps. En
outre, dans le graphique de la station au fil des ans, on constate que
la station est de nature additive puisque l'effet saisonnier fluctue
autour de zéro.

\begin{Shaded}
\begin{Highlighting}[]
\NormalTok{hawai_ets <-}\StringTok{ }\NormalTok{hawai_train }\OperatorTok\StringTok{ }\KeywordTok{ets}\NormalTok{()}
\NormalTok{hawai_ets}
\end{Highlighting}
\end{Shaded}

\begin{verbatim}
## ETS(A,A,A) 
## 
## Call:
##  ets(y = .) 
## 
##   Smoothing parameters:
##     alpha = 0.6698 
##     beta  = 0.003 
##     gamma = 2e-04 
## 
##   Initial states:
##     l = 315.2483 
##     b = 0.102 
##     s = 0.6164 0.0116 -0.9384 -2.0207 -3.0899 -2.8853
##            -1.2381 0.7047 2.2379 2.8344 2.416 1.3513
## 
##   sigma:  0.3404
## 
##      AIC     AICc      BIC 
## 1408.202 1409.940 1474.731
\end{verbatim}

\begin{Shaded}
\begin{Highlighting}[]
\KeywordTok{autoplot}\NormalTok{(hawai_ets)}
\end{Highlighting}
\end{Shaded}

\includegraphics{exercice-_5_files/figure-latex/unnamed-chunk-6-1.pdf}

\hypertarget{pruxe9vision-de-co2-atmosphuxe9rique-pour-comparer-aux-donnuxe9es-test}{%
\subsection{Prévision de CO2 atmosphérique pour comparer aux données
test}\label{pruxe9vision-de-co2-atmosphuxe9rique-pour-comparer-aux-donnuxe9es-test}}

La fonction \texttt{forecast:\ :\ forecast()} a été utilisée pour
obtenir la prédiction du modèle. L'argument h (nombre de périodes pour
les prévisions) de 156 qui correspond au nombre d'observations, qui
composent la série de test (longeur de la série de test).

Le graphique suivant présente les performances du modèle de prévision :
la ligne bleue indique les valeurs obtenues, tandis que la ligne verte
indique les valeurs obtenues à partir de la série de tests. Les régions
bleues et transparentes font référence aux intervalles prévisionnels du
modèle.

\begin{Shaded}
\begin{Highlighting}[]
\NormalTok{hawaifc <-}\StringTok{ }\NormalTok{hawai_ets }\OperatorTok\StringTok{ }\KeywordTok{forecast}\NormalTok{(}\DataTypeTok{h =} \KeywordTok{length}\NormalTok{(hawai_test))}
\KeywordTok{autoplot}\NormalTok{(hawaifc) }\OperatorTok{+}\StringTok{ }\KeywordTok{autolayer}\NormalTok{(hawai_test, }\DataTypeTok{color =} \StringTok{"green"}\NormalTok{) }
\end{Highlighting}
\end{Shaded}

\includegraphics{exercice-_5_files/figure-latex/unnamed-chunk-7-1.pdf}
On observe que l'approximation du modèle est plus élevée pour certaines
années, tandis que pour d'autres, il y a une sous-estimation ou une
surestimation, cependant, il est pertinent de faire une analyse de
précision et de résidus pour la vérifier.

Dans le tableau suivant, on observe que l'erreur d'échelle absolue
moyenne (MASE) de la prévision est de 0,7198956 et celle du modèle est
de 0,2034533, ce qui indique que le modèle fournit une bonne prévision
puisque plus cette valeur est proche de zéro, meilleure est la capacité
de prévision du modèle.

\begin{Shaded}
\begin{Highlighting}[]
\KeywordTok{accuracy}\NormalTok{(hawaifc, hawai_ts)}
\end{Highlighting}
\end{Shaded}

\begin{verbatim}
##                       ME      RMSE       MAE         MPE       MAPE      MASE
## Training set 0.009127538 0.3329635 0.2496935 0.002272597 0.07593717 0.2034533
## Test set     0.258792487 1.1125984 0.8835110 0.067546737 0.24278880 0.7198956
##                    ACF1 Theil's U
## Training set 0.06670592        NA
## Test set     0.94372350 0.8673732
\end{verbatim}

\hypertarget{analyse-des-ruxe9sidus}{%
\subsection{Analyse des résidus}\label{analyse-des-ruxe9sidus}}

Afin de vérifier ces propriétés, la fonction
\texttt{forecast::checkresiduals()} est utilisée comme suit :

\begin{Shaded}
\begin{Highlighting}[]
\KeywordTok{checkresiduals}\NormalTok{(hawai_ets)}
\end{Highlighting}
\end{Shaded}

\includegraphics{exercice-_5_files/figure-latex/unnamed-chunk-9-1.pdf}

\begin{verbatim}
## 
##  Ljung-Box test
## 
## data:  Residuals from ETS(A,A,A)
## Q* = 54.438, df = 8, p-value = 5.676e-09
## 
## Model df: 16.   Total lags used: 24
\end{verbatim}

Il est à noter que la p-value obtenue est de 5.676e-09.Il est
statistiquement significative. Elle indique une preuve solide contre
l'hypothèse nulle, car il y a moins de 5\% de probabilité que
l'hypothèse nulle soit correcte (et les résultats sont aléatoires).

En général, les graphiques ci-dessus montrent que la moyenne des résidus
est proche de zéro et qu'il n'y a pas de corrélation significative entre
eux. La variation présentée dans le graphique des résidus en fonction du
temps tend à être constante. Alors que l'histogramme des résidus indique
qu'il n'y a pas de distribution normale, la normalité des résidus sera
vérifiée par un test de Shapiro-Wilk.

\begin{Shaded}
\begin{Highlighting}[]
\KeywordTok{shapiro.test}\NormalTok{(}\KeywordTok{residuals}\NormalTok{(hawai_ets))}
\end{Highlighting}
\end{Shaded}

\begin{verbatim}
## 
##  Shapiro-Wilk normality test
## 
## data:  residuals(hawai_ets)
## W = 0.96983, p-value = 6.061e-07
\end{verbatim}

La p-value (\textasciitilde{} 6.061e-07) indique que la distribution des
résidus n'est pas normale.

Une méthode de prévision est considérée bonne lorsque ses résidus
présentent les propriétés suivantes : absence de corrélation entre les
résidus, la moyenne des résidus est de zéro, la variance des résidus est
constante, les résidus sont normalement distribués. Bien que de ces
quatre paramètres, le dernier n'est pas rempli, la méthode appliquée
génère de bonnes prévisions. Cependant, l'application d'une
transformation de Box-Cox peut parfois contribuer à améliorer la
normalité de la distribution des résidus.

\hypertarget{application-de-la-transformation-de-box-cox}{%
\subsubsection{Application de la transformation de
Box-Cox}\label{application-de-la-transformation-de-box-cox}}

La modélisation des séries temporelles avec la transformation Box-Cox
est effectuée de la même manière que pour les séries temporelles sans
transformation, mais en utilisant le paramètre lambda dans l'application
de la méthode ETS, qui est estimé avec la fonction de
prévision::BoxCox.lambda().

\begin{Shaded}
\begin{Highlighting}[]
\NormalTok{hawai_BC <-}\StringTok{ }\NormalTok{hawai_train }\OperatorTok\StringTok{ }\KeywordTok{ets}\NormalTok{(}\DataTypeTok{lambda =} \KeywordTok{BoxCox.lambda}\NormalTok{(hawai_train))}

\NormalTok{hawaifc_BC <-}\StringTok{ }\NormalTok{hawai_BC }\OperatorTok
\StringTok{  }\KeywordTok{forecast}\NormalTok{(}\DataTypeTok{h =} \KeywordTok{length}\NormalTok{(hawai_test))}

\KeywordTok{autoplot}\NormalTok{(hawaifc_BC) }\OperatorTok{+}\StringTok{ }\KeywordTok{autolayer}\NormalTok{(hawai_test, }\DataTypeTok{color =} \StringTok{"green"}\NormalTok{) }
\end{Highlighting}
\end{Shaded}

\includegraphics{exercice-_5_files/figure-latex/unnamed-chunk-11-1.pdf}

\begin{Shaded}
\begin{Highlighting}[]
\KeywordTok{accuracy}\NormalTok{(hawaifc_BC, hawai_ts)}
\end{Highlighting}
\end{Shaded}

\begin{verbatim}
##                      ME      RMSE       MAE         MPE       MAPE      MASE
## Training set 0.01317536 0.3201071 0.2414950 0.003562256 0.07340163 0.1967731
## Test set     0.17310078 1.0530018 0.8432859 0.044216546 0.23196266 0.6871197
##                    ACF1 Theil's U
## Training set 0.03928175        NA
## Test set     0.94442197 0.8221877
\end{verbatim}

\begin{Shaded}
\begin{Highlighting}[]
\KeywordTok{checkresiduals}\NormalTok{(hawai_BC)}
\end{Highlighting}
\end{Shaded}

\includegraphics{exercice-_5_files/figure-latex/unnamed-chunk-13-1.pdf}

\begin{verbatim}
## 
##  Ljung-Box test
## 
## data:  Residuals from ETS(A,A,A)
## Q* = 42.618, df = 8, p-value = 1.037e-06
## 
## Model df: 16.   Total lags used: 24
\end{verbatim}

\begin{Shaded}
\begin{Highlighting}[]
\KeywordTok{shapiro.test}\NormalTok{(}\KeywordTok{residuals}\NormalTok{(hawai_BC))}
\end{Highlighting}
\end{Shaded}

\begin{verbatim}
## 
##  Shapiro-Wilk normality test
## 
## data:  residuals(hawai_BC)
## W = 0.96958, p-value = 5.491e-07
\end{verbatim}

Après l'application de la transformation, il est possible de constater
que les paramètres statistiques ont diminué, mais pas de manière
significative. En ce qui concerne les résidus, ils ne présentent pas
d'amélioration, conservant les mêmes caractéristiques que les précédents
et la distribution des résidus n'est pas normale.

\hypertarget{conclusion}{%
\subsection{Conclusion}\label{conclusion}}

On peut voir que les données montrent une tendance à la croissance. Le
modèle appliqué montre de bonnes prévisions, il n'y a pas de corrélation
entre les résidus, leur moyenne est nulle et la variance essaie de
rester constante. Une transformation Box-Cox a été effectuée afin de
savoir si les données pouvaient avoir une distribution normale,
cependant, cela n'a pas été le cas et la modification des résultats des
autres paramètres n'a pas été significative, c'est pourquoi il est
décidé de continuer avec le modèle initialement proposé. Un autre modèle
pourrait être essayé afin de comparer les résultats obtenus.

\end{document}
